% Created 2018-12-15 Sat 00:10
% Intended LaTeX compiler: pdflatex
\ifdefined\kanjiskip
                                                   \documentclass[autodetect-engine,dvipdfmx,12pt,a4paper,ja=standard]{bxjsarticle}
                                                 \else
                                                   \ifdefined\pdfoutput
                                                     \ifnum\pdfoutput=0
                                                       \documentclass[autodetect-engine,dvipdfmx,12pt,a4paper,ja=standard]{bxjsarticle}
                                                     \else
                                                       \documentclass[autodetect-engine,12pt,a4paper,ja=standard]{bxjsarticle}
                                                     \fi
                                                   \else
                                                     \documentclass[autodetect-engine,12pt,a4paper,ja=standard]{bxjsarticle}
                                                   \fi
                                                 \fi
                                                 \usepackage{amsmath}
                                                 \usepackage{newtxtext,newtxmath}
                                                 \usepackage{graphicx}
                                                 \usepackage{hyperref}
                                                 \ifdefined\kanjiskip
                                                 \usepackage{pxjahyper}
                                                 \hypersetup{colorlinks=true}
                                                 \else
                                                 \ifdefined\XeTeXversion
                                                 \hypersetup{colorlinks=true}
                                                 \else
                                                 \ifdefined\directlua
                                                 \hypersetup{pdfencoding=auto,colorlinks=true}
                                                 \else
                                                 \hypersetup{unicode,colorlinks=true}
                                                 \fi
                                                 \fi
                                                 \fi
\author{細田弘吉}
\date{\today}
\title{}
\hypersetup{
 pdfauthor={細田弘吉},
 pdftitle={},
 pdfkeywords={},
 pdfsubject={},
 pdfcreator={Emacs 26.1 (Org mode 9.1.13)},
 pdflang={English}}
\begin{document}

\tableofcontents

\section{Posts}
\label{sec:org713ac34}
\subsection{Emacsのorg-modeで論文を書く(その5:htmlへのexportの際のフォントの色の変更などいろいろ)\hfill{}\textsc{emacs:orgmode:html:export:css:color}}
\label{sec:org31add9a}
学会発表や論文作成にあたっては,当然のことながら,その分野の他の研究者の論文を読んでまとめるなどの作業を行う.そこで,論文の要旨などをorg-modeにざっとまとめておくと,pdfにもhtmlにもexport出来て便利である.pdfは印刷に向いているが,htmlは多くの論文をいっぺんに見るのに向いており,また,compileの時間もpdfより圧倒的に速い.そこで,今回は,org-modeからhtmlへexportする際の有用な小技について書いてみたい.

\subsubsection{フォントの色の変更やハイライト}
\label{sec:org89e68a3}
\begin{enumerate}
\item \href{https://github.com/fniessen/org-macros}{Org Macros}
\label{sec:orged7a6b9}
\begin{itemize}
\item フォントの色を変更する方法はいろいろあるが,ハイライトや背景の色の変更までできるこの \href{https://github.com/fniessen/org-macros}{Org Macros} \footnote{\url{https://github.com/fniessen/org-macros}} が一番便利である.内容は,org-modeの便利ななマクロ集である.リンク先からダウンロードして,適当なところに保存し,解凍しておく.ここでは,/Users/taipapa/hoge/fuga/org-macros.setupに置くことにする.使い方は簡単で上記のwebsiteに書いてあるとおり,各org fileの先頭に以下のように記述してorg-macros.setupの場所を教えてやれば良い.
\begin{verbatim}
#+INCLUDE: /Users/taipapa/hoge/fuga/org-macros.setup
\end{verbatim}
これだけである.
\item 注意事項としては,このブログはox-hugoで書いているが,ox-hugoの場合は文書の先頭に上記を書いても効かない.各ポストのpropertyのあとに書いておけば効く.各ポストごとに設定するようになっているらしい.....(全国15人?ぐらいの人にしか意味のない注意書きである)
\item いくつか使い方の例をあげておく
\begin{verbatim}
{{{color(blue, 青くなるかな?)}}}
*{{{color(blue, ボールドで青くなるかな?)}}}*
{{{highlight(yellow, 黄色にハイライトされるかな?)}}}
*{{{highlight(yellow, 黄色にハイライトされて文字はボールドになるかな?)}}}*
{{{bgcolor(yellow, 背景が黄色になるかな?)}}}
*{{{bgcolor(yellow, 背景が黄色になって文字はボールドになるかな?)}}}*
\end{verbatim}
これが以下のように表示される.
\begin{itemize}
\item \textcolor{blue}{ 青くなるかな?}
\item \textbf{\textcolor{blue}{ ボールドで青くなるかな?}}
\item
\item \textbf{}
\item
\item \textbf{}
\end{itemize}
\item 上記以外にも多くのマクロが含まれており,そちらも人によっては有用かもしれない.
\item なお,org-modeのマクロ自体に関しては,org-modeのマニュアルの \textbf{12.5 Macro replacement} を参考にしていただきたい.
\end{itemize}
\end{enumerate}
\end{document}
